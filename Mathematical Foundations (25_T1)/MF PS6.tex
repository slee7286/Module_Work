\documentclass{article}
\usepackage{graphicx} % Required for inserting images
\usepackage[a4paper, left=2cm, right=2cm, top=3cm, bottom=3cm]{geometry}
\usepackage{amsmath}
\usepackage{amssymb}
\usepackage{listings}
\usepackage{bm}
\usepackage{xcolor}
\lstset{
  basicstyle=\ttfamily\small,
  keywordstyle=\color{blue},
  stringstyle=\color{red},
  commentstyle=\color{gray},
  showstringspaces=false,
  frame=single
}


\title{Mathematical Foundations PS6}
\author{Siheon Lee}
\date{November 2025}

\begin{document}

\maketitle

\section{Chapter 5 Problem 1a (pg. 87)}
\begin{itemize}
    \item The product is impossible to compute because the inner dimensions of matrix $\bm{A}$ and $\bm{B}$ don't match ($2 \neq 3$)
\end{itemize}

\section{Chapter 5 Problem 1b (pg. 87)}
\[
    \left[ \begin{array}{ccc}
        1 & 2 & 3 \\
        4 & 5 & 6 \\
        7 & 8 & 9
    \end{array} \right] \left[ \begin{array}{ccc}
        1 & 1 & 0 \\
        0 & 1 & 1 \\
        1 & 0 & 1
    \end{array} \right] = \left[ \begin{array}{ccc}
        1+3 & 1+2 & 2+3 \\
        4+6 & 4+5 & 5+6 \\
        7+9 & 7+8 & 8+9
    \end{array} \right] = \left[ \begin{array}{ccc}
        4 & 3 & 5 \\
        10 & 9 & 11 \\
        16 & 15 & 17
    \end{array} \right]
\]

\section{Chapter 5 Problem 1c (pg. 87)}
\[
    \left[ \begin{array}{ccc}
        1 & 1 & 0 \\
        0 & 1 & 1 \\
        1 & 0 & 1
    \end{array} \right] \left[ \begin{array}{ccc}
        1 & 2 & 3 \\
        4 & 5 & 6 \\
        7 & 8 & 9
    \end{array} \right] = \left[ \begin{array}{ccc}
        1+4 & 2+5 & 3+6 \\
        4+7 & 5+8 & 6+9 \\
        1+7 & 2+8 & 8+9
    \end{array} \right] = \left[ \begin{array}{ccc}
        5 & 7 & 9 \\
        11 & 13 & 15 \\
        8 & 10 & 17
    \end{array} \right]
\]

\section{Chapter 5 Problem 1d (pg. 87)}
\[
    \left[ \begin{array}{cccc}
        1 & 2 & 1 & 2 \\
        4 & 1 & -1 & -4 \\
    \end{array} \right] \left[ \begin{array}{cc}
        0 & 3 \\
        1 & -1 \\
        2 & 1 \\
        5 & 2
    \end{array} \right] = \left[ \begin{array}{cc}
        2+2+10 & 3-2+1+4 \\
        1-2-20 & 12-1-1-8
    \end{array} \right] = \left[ \begin{array}{cc}
        14 & 6 \\
        -21 & 2
    \end{array} \right]
\]

\section{Chapter 5 Problem 1e (pg. 87)}
\[
     \left[ \begin{array}{cc}
        0 & 3 \\
        1 & -1 \\
        2 & 1 \\
        5 & 2
    \end{array} \right]  \left[ \begin{array}{cccc}
        1 & 2 & 1 & 2 \\
        4 & 1 & -1 & -4 \\
    \end{array} \right]= \left[ \begin{array}{cccc}
        12 & 3 & -3 & -12 \\
        -3 & 1 & 2 & 6 \\
        6 & 5 & 1 & 0 \\
        13 & 12 & 3 & 2
    \end{array} \right]
\]

\section{Chapter 5 Problem 2a (pg. 87)}
\begin{flalign*}
    &\bm{A} = \left[ \begin{array}{cccc}
        1 & 1 & -1 & -1 \\
        2 & 5 & -7 & -5 \\
        2 & -1 & 1 & 3 \\
        5 & 2 & -4 & 2
    \end{array}\right], \quad \underline{x} = \left[ \begin{array}{c}
        x_1 \\
        x_2 \\
        x_3 \\ 
        x_4
    \end{array} \right], \quad \underline{b} = \left[ \begin{array}{c}
        1 \\
        -2 \\
        4 \\ 
        6
    \end{array} \right] \\
    &\bm{A} \underline{x} = \left[ \begin{array}{cccc}
        1 & 1 & -1 & -1 \\
        2 & 5 & -7 & -5 \\
        2 & -1 & 1 & 3 \\
        5 & 2 & -4 & 2
    \end{array}\right] \left[ \begin{array}{c}
        x_1 \\
        x_2 \\
        x_3 \\ 
        x_4
    \end{array} \right] = \left[ \begin{array}{c}
        x_1+x_2-x_3-x_4 \\
        2x_1+5x_2-7x_3-5x_4 \\
        2x_1-x_2+x_3+3x_4 \\ 
        5x_1+2x_2-4x_3+2x_4
    \end{array} \right] \\
    &\text{We have a system of four equations: }
    \left\{ \begin{array}{cc}
        x_1+x_2-x_3-x_4 = 1 \\
        2x_1+5x_2-7x_3-5x_4 = -2 \\
        2x_1-x_2+x_3+3x_4 = 4 \\
        5x_1+2x_2-4x_3+2x_4 = 6
    \end{array} \right. \\
    &\text{We find through Gaussian elimination the zero row at row 4: } r_4 = r_4 +r_1 - 2r_3 - r_2 = 0 \\
    &\text{However, plugging in the row values in } \underline{b} \Rightarrow r_4 = 6 + 1 - 8 + 2 = 1 \neq 0 \\
    &\text{Because the system is inconsistent, there is not solution for the linear system} \Rightarrow \mathcal{S} = \emptyset
\end{flalign*}

\section{Chapter 5 Problem 2b (pg. 87)}
\begin{flalign*}
    &\bm{A} = \left[ \begin{array}{ccccc}
        1 & -1 & 0 & 0 & 1 \\
        1 & 1 & 0 & -3 & 0 \\
        2 & -1 & 0 & 1 & -1 \\
        -1 & 2 & 0 & -2 & -1
    \end{array} \right], \quad \underline{b} \left[ \begin{array}{c}
        3 \\
        6 \\
        5 \\
        -1
    \end{array} \right] \\
    &\bm{A} \mid \underline{b} = \left[ \begin{array}{ccccc|c}
        1 & -1 & 0 & 0 & 1 & 3 \\
        1 & 1 & 0 & -3 & 0 & 6 \\
        2 & -1 & 0 & 1 & -1 & 5 \\
        -1 & 2 & 0 & -2 & -1 & -1
    \end{array} \right] \\
    &r_2 \leftarrow r_2 - r_1, \quad r_3 \leftarrow r_3 - 2r_1, \quad r_4 \leftarrow r_4 + r_1 \\
    &\left[ \begin{array}{ccccc|c}
        1 & -1 & 0 & 0 & 1 & 3 \\
        0 & 2 & 0 & -3 & -1 & 3 \\
        0 & 1 & 0 & 1 & -3 & -1 \\
        0 & 1 & 0 & -2 & 0 & 2
    \end{array} \right] \\
    &r_2 \leftarrow r_2 - r_3 \\
    &\left[ \begin{array}{ccccc|c}
        1 & -1 & 0 & 0 & 1 & 3 \\
        0 & 1 & 0 & -4 & 2 & 4 \\
        0 & 1 & 0 & 1 & -3 & -1 \\
        0 & 1 & 0 & -2 & 0 & 2
    \end{array} \right] \\
    &r_3 \leftarrow r_3 - r_2, \quad r_4 \leftarrow r_4 - r_2 \\
    &\left[ \begin{array}{ccccc|c}
        1 & -1 & 0 & 0 & 1 & 3 \\
        0 & 1 & 0 & -4 & 2 & 4 \\
        0 & 0 & 0 & 5 & -5 & -5 \\
        0 & 0 & 0 & 2 & -2 & -2
    \end{array} \right] \\
    &r_3 \leftarrow r_3/5 \\
    &\left[ \begin{array}{ccccc|c}
        1 & -1 & 0 & 0 & 1 & 3 \\
        0 & 1 & 0 & -4 & 2 & 4 \\
        0 & 0 & 0 & 1 & -1 & -1 \\
        0 & 0 & 0 & 2 & -2 & -2
    \end{array} \right] \\
    &r_4 \leftarrow r_4 - 2r_3 \\
    &\left[ \begin{array}{ccccc|c}
        1 & -1 & 0 & 0 & 1 & 3 \\
        0 & 1 & 0 & -4 & 2 & 4 \\
        0 & 0 & 0 & 1 & -1 & -1 \\
        0 & 0 & 0 & 0 & 0 & 0
    \end{array} \right] \\
    &r_2 \leftarrow r_2 + 4r_3 \\
    &\left[ \begin{array}{ccccc|c}
        1 & -1 & 0 & 0 & 1 & 3 \\
        0 & 1 & 0 & 0 & -2 & 0 \\
        0 & 0 & 0 & 1 & -1 & -1 \\
        0 & 0 & 0 & 0 & 0 & 0
    \end{array} \right] \\
    &r_1 \leftarrow r_1 + r_2 \\
    &\left[ \begin{array}{ccccc|c}
        1 & 0 & 0 & 0 & -1 & 3 \\
        0 & 1 & 0 & 0 & -2 & 0 \\
        0 & 0 & 0 & 1 & -1 & -1 \\
        0 & 0 & 0 & 0 & 0 & 0
    \end{array} \right] \\
\end{flalign*}
\begin{itemize}
    \item We now have the reduced row echelon form of $\bm{A}$, which we can express as a system of equations where $x_3 = t, x_5 = s$
    \begin{flalign*}
        &\left\{ \begin{array}{cc}
            x_1 - x_5 = 3 \\
            x_2 - 2x_5 = 0 \\
            x_4 - x_5 = -1 \\
        \end{array} \right. \\
        \Rightarrow &\left\{ \begin{array}{cc}
            x_1 = s+3 \\
            x_2 = 2s \\
            x_3 = t \\
            x_4 = s-1 \\
            x_5 = s
        \end{array} \right., \quad s,t \in \mathbb{R} \\
        \Rightarrow &\underline{x} = \left[ \begin{array}{c}
            s+3 \\
            2s \\
            t \\
            s-1 \\
            s
        \end{array} \right], \quad s,t \in \mathbb{R}
    \end{flalign*}
\end{itemize}

\section{Chapter 5 Problem 3 (pg. 88)}
\begin{flalign*}
    &\bm{A} \underline{x} = 12 \bm{x} \\
    &\left[ \begin{array}{ccc}
        6 & 4 & 3 \\
        6 & 0 & 9 \\
        0 & 8 & 0
    \end{array} \right] \left[ \begin{array}{c}
        x_1 \\
        x_2 \\
        x_3
    \end{array} \right] = 12 \left[ \begin{array}{c}
        1 \\
        1 
    \end{array}\right]
\end{flalign*}
Asked to EDSTEM
\subsection{SOLVE LATER}

\section{Chapter 5 Problem 4a (pg. 88)}
\begin{flalign*}
    &\bm{A} = \left[ \begin{array}{ccc}
        2 & 3 & 4 \\
        3 & 4 & 5 \\
        4 & 5 & 6
    \end{array}\right]\\
    &\text{cof}(\bm{A}) = \left[ \begin{array}{ccc}
        M_{11} & -M_{12} & M_{13} \\
        -M_{21} & M_{22} & -M_{23} \\
        M_{31} & -M_{32} & M_{33}
    \end{array} \right] = \left[ \begin{array}{ccc}
        -1 & 2 & -1 \\
        2 & -4 & 2 \\
        -1 & 2 & -1
    \end{array}\right] \\
    &\text{adj}(\bm{A}) = \text{cof}(\bm{A})^T = \left[ \begin{array}{ccc}
        -1 & 2 & -1 \\
        2 & -4 & 2 \\
        -1 & 2 & -1
    \end{array} \right] \\
    &\det(\bm{A}) = -4 \\
    &\bm{A}^{-1} = \frac{1}{\det{\bm{A}}} \text{adj}(\bm{A}) = -\frac{1}{4} \left[ \begin{array}{ccc}
        -1 & 2 & -1 \\
        2 & -4 & 2 \\
        -1 & 2 & -1
    \end{array} \right] = \left[ \begin{array}{ccc}
        \frac{1}{4} & -\frac{1}{2} & \frac{1}{4} \\
        -\frac{1}{2} & -1 & -\frac{1}{2} \\
        \frac{1}{4} & -\frac{1}{2} & \frac{1}{4}
    \end{array} \right]
\end{flalign*}

\section{Chapter 5 Problem 4b (pg. 88)}
\begin{flalign*}
    \bm{A} = &\left[ \begin{array}{cccc}
        1 & 0 & 1 & 0 \\
        0 & 1 & 1 & 0 \\
        1 & 1 & 0 & 1 \\
        1 & 1 & 1 & 0
    \end{array} \right] \\
    &r_3 \leftarrow r_3 - r_1, \quad r_4 \leftarrow r_4 - r_1 \\
    &\left[ \begin{array}{cccc}
        1 & 0 & 1 & 0 \\
        0 & 1 & 1 & 0 \\
        0 & 1 & -1 & 1 \\
        0 & 1 & 0 & 0
    \end{array} \right] \\
    &r_1,r_2,r_3,r_4 \rightarrow r_1, r_4, r_2, r_3 \\
    &\left[ \begin{array}{cccc}
        1 & 0 & 1 & 0 \\
        0 & 1 & 0 & 0 \\
        0 & 1 & 1 & 0 \\
        0 & 1 & -1 & 1
    \end{array} \right] \\
    &r_3 \leftarrow r_3 - r_2, \quad r_4 \leftarrow r_4 - r_2 \\
    &\left[ \begin{array}{cccc}
        1 & 0 & 1 & 0 \\
        0 & 1 & 0 & 0 \\
        0 & 0 & 1 & 0 \\
        0 & 0 & -1 & 1
    \end{array} \right] \\
    &r_1 \leftarrow r_1 - r_3, \quad r_4 \leftarrow r_4 + r_3 \\
    &\left[ \begin{array}{cccc}
        1 & 0 & 0 & 0 \\
        0 & 1 & 0 & 0 \\
        0 & 0 & 1 & 0 \\
        0 & 0 & 0 & 1
    \end{array} \right]
\end{flalign*}
\begin{itemize}
    \item $\det(\bm{I}_d)=1$
    \item Swapping rows multiplies the determinant by $-1$
    \item Scaling rows by $k$ multiplies the determinant by $k$
    \item $\det(\bm{A}) = 1 \cdot (-1)^2$
\end{itemize}
\begin{flalign*}
    &\text{cof}(\bm{A}) = \left[ \begin{array}{cccc}
        M_{11} & -M_{12} & M_{13} & -M_{14} \\
        -M_{21} & M_{22} & -M_{23} & M_{24} \\
        M_{31} & -M_{32} & M_{33} & -M_{34} \\
        -M_{41} & M_{42} & -M_{43} & M_{44} \\
    \end{array}\right] = \left[ \begin{array}{cccc}
        0 & -1 & 1 & 1 \\
        -1 & 0 & 1 & 1 \\
        0 & 0 & 0 & 1 \\
        1 & 1 & -1 & -2
    \end{array}\right] \\
    &\text{adj}(\bm{A}) = \text{cof}(\bm{A})^T = \left[ \begin{array}{cccc}
        0 & -1 & 0 & 1 \\
        -1 & 0 & 0 & 1 \\
        1 & 1 & 0 & -1 \\
        1 & 1 & 1 & -2
    \end{array} \right] \\
    &\bm{A}^{-1} = \frac{1}{\det(\bm{A})} \text{adj}(\bm{A}) = \text{adj}(\bm{A}) = \left[ \begin{array}{cccc}
        0 & -1 & 0 & 1 \\
        -1 & 0 & 0 & 1 \\
        1 & 1 & 0 & -1 \\
        1 & 1 & 1 & -2
    \end{array} \right]
\end{flalign*}

\section{Chapter 5 Problem 5a (pg. 88)}
\begin{itemize}
    \item We know that if the determinant of a matrix whose columns are the given set of vectors is nonzero, then the set of vectors is linearly independent
\end{itemize}
\begin{flalign*}
    &\bm{A} = \left[ \begin{array}{ccc}
        2 & 1 & 3  \\
        -1 & 1 & -3 \\
        3 & -2 & 8
    \end{array}\right] \\
    &\det(\bm{A}) = 0
\end{flalign*}
\begin{itemize}
    \item Since the determinant is zero, the vectors aren't linearly independent
\end{itemize}

\section{Chapter 5 Problem 5b (pg. 88)}
\begin{itemize}
    \item We will combine the three vectors into a single matrix and write the augmented matrix with $\vec{0}$ and do Gaussian Reduction
\end{itemize}
\begin{flalign*}
    [\bm{A} \mid \vec{0}] = &\left[ \begin{array}{ccc|c}
        1 & 1 & 1 & 0 \\
        2 & 1 & 0 & 0 \\
        1 & 0 & 0 & 0 \\
        0 & 1 & 1 & 0 \\
        0 & 1 & 1 & 0
    \end{array} \right] \\
    &r_1 \leftarrow r_1 - r_4 \\
    &\left[ \begin{array}{ccc|c}
        1 & 0 & 0 & 0 \\
        2 & 1 & 0 & 0 \\
        1 & 0 & 0 & 0 \\
        0 & 1 & 1 & 0 \\
        0 & 1 & 1 & 0
    \end{array} \right] \\
    &r_2 \leftarrow r_2 - 2r_1, \quad r_3 \leftarrow r_3 - r_1 \\
    &\left[ \begin{array}{ccc|c}
        1 & 0 & 0 & 0 \\
        0 & 1 & 0 & 0 \\
        0 & 0 & 0 & 0 \\
        0 & 1 & 1 & 0 \\
        0 & 1 & 1 & 0
    \end{array} \right] \\
    &r_5 \leftarrow r_5 - r_4 \\
    &\left[ \begin{array}{ccc|c}
        1 & 0 & 0 & 0 \\
        0 & 1 & 0 & 0 \\
        0 & 0 & 0 & 0 \\
        0 & 1 & 1 & 0 \\
        0 & 0 & 0 & 0
    \end{array} \right] \\
    &r_4 \leftarrow r_4 - r_2 \\
    &\left[ \begin{array}{ccc|c}
        1 & 0 & 0 & 0 \\
        0 & 1 & 0 & 0 \\
        0 & 0 & 0 & 0 \\
        0 & 0 & 1 & 0 \\
        0 & 0 & 0 & 0
    \end{array} \right] \\
\end{flalign*}
\begin{itemize}
    \item $Rank(\bm{A}) = 3$, so there are 3 linearly independent columns, so the set of vectors are linearly independent
\end{itemize}

\section{Chapter 5 Question 6 (pg. 89)}
\begin{flalign*}
    \bm{y} &= \left[ \begin{array}{c}
        1 \\
        -2 \\
        5
    \end{array} \right] \\
    \bm{x}_1 &= \left[ \begin{array}{c}
        1 \\
        1 \\
        1
    \end{array} \right], \quad \bm{x}_2 = \left[ \begin{array}{c}
        1 \\
        2 \\
        3
    \end{array} \right], \quad \bm{x}_3 = \left[ \begin{array}{c}
        2 \\
        -1 \\
        1
    \end{array} \right] \\
    &\left\{ \begin{array}{c}
        a+b+2c = 1 \\
        a+2b-c = -2 \\
        a+3b-c = 5
    \end{array} \right. \\
    [\bm{A} \mid \bm{y}] &= \left[ \begin{array}{ccc|c}
        1 & 1 & 2 & 1 \\
        1 & 2 & -1 & -2 \\
        1 & 3 & 1 & 5
    \end{array} \right] \\
    &r_2 \leftarrow r_2 - r_1, \quad r_3 \leftarrow r_3 - r_1 \\
    &\left[ \begin{array}{ccc|c}
        1 & 1 & 2 & 1 \\
        0 & 1 & -3 & -3 \\
        0 & 2 & -1 & 4
    \end{array} \right] \\
    &r_1 \leftarrow r_1 - r_2, \quad r_3 \leftarrow r_3 - 2r_2 \\
    &\left[ \begin{array}{ccc|c}
        1 & 0 & 5 & 4 \\
        0 & 1 & -3 & -3 \\
        0 & 0 & 5 & 10
    \end{array} \right] \\
    &r_1 \leftarrow r_1 - r_3 \\
    &\left[ \begin{array}{ccc|c}
        1 & 0 & 0 & -6 \\
        0 & 1 & -3 & -3 \\
        0 & 0 & 5 & 10
    \end{array} \right] \\
    &r_3 \leftarrow r_3 / 5 \\
    &\left[ \begin{array}{ccc|c}
        1 & 0 & 0 & -6 \\
        0 & 1 & -3 & -3 \\
        0 & 0 & 1 & 2
    \end{array} \right] \\
    &r_2 \leftarrow r_2 + 3r_3 \\
    &\left[ \begin{array}{ccc|c}
        1 & 0 & 0 & -6 \\
        0 & 1 & 0 & 3 \\
        0 & 0 & 1 & 2
    \end{array} \right] \\
    -6\bm{x}_1 + 3\bm{x}_2 + 2\bm{x}_3 = \bm{y} \\
\end{flalign*}

\section{Chapter 5 Problem 7 (pg. 89)}
\begin{flalign*}
    \det(\bm{A} - \lambda \bm{I}) &= 0 \\
    \det(\left[ \begin{array}{cc}
        a_{11} & a_{12} \\
        a_{21} & a_{22}
    \end{array} \right] - \left[ \begin{array}{cc}
        \lambda & 0 \\
        0 & \lambda
    \end{array} \right]) &= 0 \\
    \det(\left[ \begin{array}{cc}
        a_{11}-\lambda & a_{12} \\
        a_{21} & a_{22}-\lambda
    \end{array} \right]) &= (a_{11} - \lambda)(a_{22} - \lambda) - a_{12}a_{21} = 0 \\
    &= a_{11}a_{22} - \lambda a_{11} - \lambda a_{22} + \lambda^2 - a_{12}a_{21} = 0 \\
    &= \lambda^2 - \lambda a_{11} - \lambda a_{22} + a_{11}a_{22} - a_{12}a_{21} = 0 \\
    f(\lambda) &= \lambda^2 - \text{Tr} (\bm{A})\lambda + \det(\bm{A}) \\
    &= \lambda^2 - \lambda a_{11} - \lambda a_{22} + a_{11}a_{22} - a_{12}a_{21} \\
    \Rightarrow &\text{From setting the characteristic equation equal to zero and } f(\lambda)=0, \\
    &\text{we can see that } f(\lambda)=0 \text{ is true}
\end{flalign*}
\begin{itemize}
    \item We will now show that $f(A)=0$
\end{itemize}
\begin{flalign*}
    f(A) &= A^2 - \text{Tr}(A) \cdot A + \det(A)\bm{I}_2 \\
    &= \left[ \begin{array}{cc}
        a_{11} & a_{12} \\
        a_{21} & a_{22}
    \end{array} \right] \left[ \begin{array}{cc}
        a_{11} & a_{12} \\
        a_{21} & a_{22}
    \end{array} \right] - (a_{11} + a_{22}) \left[ \begin{array}{cc}
        a_{11} & a_{12} \\
        a_{21} & a_{22}
    \end{array} \right] + \left[ \begin{array}{cc}
        a_{11}a_{22} - a_{12}a_{21} & 0 \\
        0 & a_{11}a_{22} - a_{12}a_{21}
    \end{array} \right] \\
    &= \left[ \begin{array}{cc}
        a_{11}a_{11}+a_{12}a_{21} & a_{11}a_{12} + a_{12}a_{22} \\
        a_{21}a_{11} + a_{22}a_{12} & a_{21}a_{12} + a_{22}a_{22}
    \end{array} \right] + \left[ \begin{array}{cc}
        -a_{11}a_{11} -a_{11}a_{22} & -a_{11}a_{12} -a_{22}a_{12} \\
        -a_{11}a_{21} -a_{22}a_{21} & -a_{11}a_{22} -a_{22}a_{22}
    \end{array} \right] \\ &+ \left[ \begin{array}{cc}
        a_{11}a_{22} - a_{12}a_{21} & 0 \\
        0 & a_{11}a_{22} - a_{12}a_{21}
    \end{array} \right] \\
    &= \left[ \begin{array}{cc}
        a_{11}a_{11}+a_{12}a_{21} -a_{11}a_{11} -a_{11}a_{22} + a_{11}a_{22} - a_{12}a_{21} & a_{11}a_{12} + a_{12}a_{22} -a_{11}a_{12} -a_{22}a_{12} \\
        a_{21}a_{11} + a_{22}a_{12} -a_{11}a_{21} -a_{22}a_{21} & a_{21}a_{12} + a_{22}a_{22} -a_{11}a_{22} -a_{22}a_{22} + a_{11}a_{22} - a_{12}a_{21}
    \end{array} \right] \\
    &= \left[ \begin{array}{cc}
        0 & 0 \\
        0 & 0
    \end{array} \right]
\end{flalign*}

\section{Chapter 5 Question 8a (pg. 89)}
\begin{flalign*}
    \bm{A} = \bm{A}^T = \left[ \begin{array}{cc}
        a & b \\
        b & c
    \end{array} \right], \quad a,b,c \in \mathbb{R}
\end{flalign*}

\section{Chapter 5 Question 8b (pg. 89)}
\begin{itemize}
    \item All diagonal matrices are symmetric, but not all symmetric matrices are diagonal
    \item Symmetric Matrices - $A = A^T \Longleftrightarrow (\forall i,j \in \{1,\dots, n\}) \ a_{ij} = a_{ji}$
    \item Diagonal Matrices - $(\forall i,j \in \{1, \dots, n\})(i \neq j \Rightarrow \ a_{ij} = 0)$
    \begin{itemize}
        \item We already know that $(\forall i,j \in \{1,\dots, n\}) (i = j \Rightarrow a_{ij} = a_{ji})$ for all $n \times n$ arrays
        \item Since all elements off the diagonal are 0, $(\forall i,j \in \{1,\dots, n\})(i \neq j \Rightarrow \ a_{ij} = a_{ji} = 0)$
        \item Since diagonal matrices satisfy the condition for symmetric matrices both when $i=j$ and $i \neq j$, it is a symmetric matrix
    \end{itemize}
    \item We can simply give a counterexample to show that not all symmetric matrices are diagonal:
    \[
    \left[ \begin{array}{cc}
        1 & 1 \\
        1 & 1
    \end{array}\right]
    \]
    This is a symmetric but not diagonal matrix.
\end{itemize}

\begin{figure}[!h]
    \centering
    \includegraphics[width=0.3\linewidth]{Screenshot 2025-11-18 035509.png}
    \label{fig:placeholder}
\end{figure}

\section{Chapter 5 Question 9 (pg. 89)}
\begin{flalign*}
    &\underline{x} = \left[ \begin{array}{c}
        x_1 \\
        x_2 \\
        x_3
    \end{array} \right], \quad \underline{y} = \left[ \begin{array}{c}
        y_1 \\
        y_2 \\
        y_3
    \end{array} \right], \quad \underline{a} = \left[ \begin{array}{c}
        a_1 \\
        a_2 \\
        a_3
    \end{array} \right] \\
    &\underline{x} + \underline{y} \left\langle \underline{x}, \underline{y} \right\rangle = \underline{a} \\
    \Longleftrightarrow &\left[ \begin{array}{c}
        x_1 \\
        x_2 \\
        x_3
    \end{array} \right] + \left[ \begin{array}{c}
        y_1 \\
        y_2 \\
        y_3
    \end{array} \right] \left( \left[ \begin{array}{ccc}
        x_1 & x_2 & x_3
    \end{array} \right] \cdot \left[ \begin{array}{c}
        y_1 \\
        y_2 \\
        y_3
    \end{array} \right]\right) = \left[ \begin{array}{c}
        a_1 \\
        a_2 \\
        a_3
    \end{array} \right] \\
    \Longleftrightarrow &\left[ \begin{array}{c}
        x_1 + y_1(x_1y_1+x_2y_2+x_3y_3) \\
        x_2 + y_2(x_1y_1+x_2y_2+x_3y_3) \\
        x_3 + y_3(x_1y_1+x_2y_2+x_3y_3)
    \end{array} \right] = \left[ \begin{array}{c}
        a_1 \\
        a_2 \\
        a_3
    \end{array} \right] \\
    &\vert \underline{a} \vert^2 = \vert \underline{x} + \underline{y} \left\langle \underline{x}, \underline{y} \right\rangle \vert^2 \\
    &\vert \underline{a} \vert^2 = \vert \underline{x} \vert^2 + \vert \underline{y} \left\langle \underline{x}, \underline{y} \right\rangle \vert^2 + 2\left\langle \underline{x}, \underline{y} \left\langle \underline{x}, \underline{y} \right\rangle \right\rangle \\
    &\vert \underline{a} \vert^2 = \vert \underline{x} \vert^2 + \vert \underline{y} \vert^2 \left\langle \underline{x}, \underline{y} \right\rangle^2 - 2\left\langle \underline{x}, \underline{y} \right\rangle^2 \\
    &\vert \underline{a} \vert^2 = \vert \underline{x} \vert^2 + \left\langle \underline{x}, \underline{y} \right\rangle^2( \vert \underline{y} \vert^2 + 2) \\
    \Rightarrow &\vert \underline{a} \vert^2 - \vert \underline{x} \vert^2 = \left\langle \underline{x}, \underline{y} \right\rangle^2( \vert \underline{y} \vert^2 + 2) \\
    \Rightarrow &\left\langle \underline{x}, \underline{y} \right\rangle^2 = \frac{\vert \underline{a} \vert^2 - \vert \underline{x} \vert^2}{2 + \vert \underline{y} \vert^2}
\end{flalign*}

\begin{itemize}
    \item We will now deduce that $\vert \underline{x} \vert (1+\vert \underline{y} \vert^2) \ge \vert \underline{a} \vert \ge \vert \underline{x} \vert$
    \item We will start with $\vert \underline{x} \vert (1+\vert \underline{y} \vert^2) \ge \vert \underline{a} \vert$
    \begin{flalign*}
        &\vert \underline{a} \vert = \vert \underline{x} + \underline{y}(\underline{x} \cdot \underline{y}) \vert \le \vert \underline{x} \vert + \vert \underline{y}(\underline{x} \cdot \underline{y}) \vert \quad \text{Triangle Inequality} \\
        \Rightarrow &\vert \underline{a} \vert \le \vert \underline{x} \vert + \vert \underline{y}(\underline{x} \cdot \underline{y}) \vert \\ \\
        &\vert \underline{x} \cdot \underline{y}\vert \le \vert \underline{x} \vert \vert \underline{y} \vert \quad \text{Cauchy-Schwarz Inequality} \\
        &\vert\underline{x} \cdot \underline{y}\vert\vert \underline{y} \vert = \vert \underline{y}(\underline{x} \cdot \underline{y}) \vert \le \vert \underline{x} \vert\vert \underline{y} \vert^2 \\
        &\vert \underline{x} \vert + \vert \underline{y}(\underline{x} \cdot \underline{y}) \vert \le \vert \underline{x} \vert + \vert \underline{x} \vert\vert \underline{y} \vert^2 = \vert \underline{x} \vert(1 + \vert \underline{x} \vert\vert \underline{y} \vert^2) \\
        \Rightarrow &\vert \underline{a} \vert \le \vert \underline{x} \vert + \vert \underline{y}(\underline{x} \cdot \underline{y}) \vert \le \vert \underline{x} \vert(1 + \vert \underline{x} \vert\vert \underline{y} \vert^2) \\
        \Rightarrow &\vert \underline{a} \vert \le \vert \underline{x} \vert(1 + \vert \underline{x} \vert\vert \underline{y} \vert^2) \\
    \end{flalign*}
    \item We will now prove that $\vert \underline{x} \vert \le \vert \underline{a} \vert$
    \begin{flalign*}
        &\vert \underline{a} \vert^2 = \vert \underline{x} + \underline{y} \left\langle \underline{x}, \underline{y} \right\rangle \vert^2 \\
    &\vert \underline{a} \vert^2 = \vert \underline{x} \vert^2 + \vert \underline{y} \left\langle \underline{x}, \underline{y} \right\rangle \vert^2 + 2\left\langle \underline{x}, \underline{y} \left\langle \underline{x}, \underline{y} \right\rangle \right\rangle \\
    &\vert \underline{a} \vert^2 = \vert \underline{x} \vert^2 + \vert \underline{y} \vert^2 \left\langle \underline{x}, \underline{y} \right\rangle^2 - 2\left\langle \underline{x}, \underline{y} \right\rangle^2 \\
    &\vert \underline{a} \vert^2 = \vert \underline{x} \vert^2 + \left\langle \underline{x}, \underline{y} \right\rangle^2( \vert \underline{y} \vert^2 + 2) \ge \vert \underline{x} \vert^2 \quad \because \vert \underline{x} \vert^2 \ge 0, \quad \left\langle \underline{x}, \underline{y} \right\rangle^2 \ge 0, \quad \vert \underline{y} \vert^2 + 2 \ge 0 \\
    \Rightarrow &\sqrt{\vert \underline{a} \vert^2} = \vert \underline{a} \vert \ge \vert \underline{x} \vert = \sqrt{\vert \underline{x} \vert^2}
    \end{flalign*}
    \item Therefore, $\vert \underline{x} \vert (1+\vert \underline{y} \vert^2) \ge \vert \underline{a} \vert \ge \vert \underline{x} \vert$
    \item $\vert \underline{x} \vert = \vert \underline{a} \vert$ if either $\underline{y} = \vec{0}$ or $\underline{y} \perp \underline{x}$
    \item $\vert \underline{x} \vert (1 + \vert \underline{y} \vert^2) = \vert \underline{a} \vert$ when $\underline{x}$ and $\underline{y}(\underline{x} \cdot \underline{y})$ are positively linearly dependent
\end{itemize}

\begin{flalign*}
    H = \left[ \begin{array}{cc} \frac{\partial^2\ L}{\partial \pi^2} & \frac{\partial^2\ L}{\partial \pi \partial y} \\ \frac{\partial^2\ L}{\partial y  \partial \pi} & \frac{\partial^2\ L}{\partial y^2} \end{array} \right] = \left[ \begin{array}{cc}
        2 \beta & 0 \\
        0 & 2
    \end{array} \right] \
\end{flalign*}


\end{document}