\documentclass{article}
\usepackage{graphicx} % Required for inserting images
\usepackage[a4paper, left=2cm, right=2cm, top=3cm, bottom=3cm]{geometry}
\usepackage{amsmath}
\usepackage{amssymb}


\title{Mathematical Foundations PS1}
\author{Siheon Lee}
\date{October 2025}

\begin{document}

\maketitle

\section{2.3 Exercises on Relations and Logic}
\begin{itemize}
    \item 1
    \begin{itemize}
        \item $2*2=4$, $2.6*2.6=6.76$, $2.64*2.64=6.9696$, $2.645*2.645=6.996025$
    \end{itemize}
    \item 2
    \begin{itemize}
        \item print('Enter your natural number:')
number = int(input())
if number < 1:
    raise Exception("Sorry, only natural numbers")

print('Enter your number of decimal places (>=1):')
decimal = int(input())
if decimal < 1:
    raise Exception("Sorry, only decimal places greater than 0")

sqrt = 0

for i in range(number // 2 + 1):
    if i ** 2 > number:
        sqrt = i - 1
        break
    elif i ** 2 == number:
        sqrt = i
        break
else:
    sqrt = number // 2

for i in range(1, decimal + 1):
    for j in range(10):
        if (sqrt + j * 10**(-i))**2 > number:
            sqrt = sqrt + (j - 1) * 10**(-i)
            break
    else:
        sqrt = sqrt + 9 * 10**(-i)
print(f"\textbackslash nApproximate sqrt({number}) $\approx$ {round(sqrt, decimal)}")
    \end{itemize}
    \item 3 
    \begin{itemize}
        \item Proof by contradiction
        \begin{itemize}
            \item Proposition: $\sqrt{3} \notin \mathbb{Q}$
            \item Assume $\sqrt{3}$ is rational
            \begin{itemize}
                \item $(\exists p,q \in \mathbb{N})\left( HCF(p,q)=1 \wedge \sqrt{3} = \frac{p}{q} \right)$
            \end{itemize}
            \item We can see that $(\exists p,q \in \mathbb{N})\left( HCF(p,q)=1 \wedge 3q^2 = p^2 \right)$
            \item Then $p^2 \mid 3$ so $p \mid 3$
            \begin{itemize}
                \item $(\exists r \in \mathbb{N}) (3r=p)$
            \end{itemize}
            \item But $3q^2=p^2$, and so
            \begin{itemize}
                \item $3q^2 = 9r^2 \Rightarrow q^2 = 3r^2$
            \end{itemize}
            \item We see that $q^2 \mid 3$, which implies $q \mid 3$.
            \item Therefore, $p,q \mid 3$, which violates our initial assumption
            \item Therefore, our initial assumption is false and $\sqrt{3}$ is not a rational number
        \end{itemize}
    \end{itemize}
    \item 4
    \begin{itemize}
        \item Proof by contradiction
        \begin{itemize}
            \item Proposition: For all primes $p, \sqrt{p} \notin \mathbb{Q}$
            \item Assume $\sqrt{p}$ is rational
            \begin{itemize}
                \item $(\exists h,q \in \mathbb{N})\left( HCF(h,q)=1 \wedge \sqrt{p} = \frac{h}{q} \right)$
            \end{itemize}
            \item We can see that $(\exists h,q \in \mathbb{N})\left( HCF(h,q)=1 \wedge pq^2 = h^2 \right)$
            \item Then $h^2 \mid p$ so $h \mid p$
            \begin{itemize}
                \item $(\exists r \in \mathbb{N}) (pr=h)$
            \end{itemize}
            \item But $pq^2=h^2$, and so
            \begin{itemize}
                \item $pq^2 = p^2r^2 \Rightarrow q^2 = pr^2$
            \end{itemize}
            \item We see that $q^2 \mid p$, which implies $q \mid p$.
            \item Therefore, $h,q \mid p$, which violates our initial assumption
            \item Therefore, our initial assumption is false and $\sqrt{p}$ is not a rational number
        \end{itemize}
    \end{itemize}
    \item 5a
    \begin{itemize}
        \item $\varphi =  \frac{\frac{b}{a}(a+b)}{\frac{b}{a} a} = \frac{b+\frac{b^2}{a}}{b}= 1+\frac{b}{a}=1+\frac{1}{\varphi}$
        \item $\varphi^2 = \varphi^2 = (1+\frac{1}{\varphi})\varphi = \varphi + 1$
        \item $\varphi = \frac{-1 \pm \sqrt{1-4(-1)(1)}}{-2}=\frac{-1\pm\sqrt{5}}{-2}=\frac{1+\sqrt{5}}{2}$
    \end{itemize}
    \item 5b
    \begin{itemize}
        \item Proposition: $\varphi \notin \mathbb{Q}$
            \item Assume $\varphi$ is rational
            \begin{itemize}
                \item $(\exists p,q \in \mathbb{N}, p>q>0)\left( HCF(p,q)=1 \wedge \varphi = \frac{p}{q} = \frac{p+q}{p} \right)$
            \end{itemize}
            \item We can see that $(\exists p,q \in \mathbb{N})\left( HCF(p,q)=1 \wedge p^2 = q^2+pq = q(p+q \right)$
            \item Then $p^2 \mid q$ so $p \mid q$
            \item Therefore, $p \mid q$, which violates our initial assumption
            \item Therefore, our initial assumption is false and $\varphi$ is not a rational number
    \end{itemize}
    \item 5c
    \begin{itemize}
        \item $\varphi = 1+\frac{1}{\varphi} = 1+\frac{1}{1+\frac{1}{\varphi}} = 1+\frac{1}{1+\frac{1}{\varphi}} = 1+\frac{1}{1+\frac{1}{1+\frac{1}{\varphi}}} = \dots = 1+ \frac{1}{1+\frac{1}{1+\frac{1}{1+\ddots}}}$
    \end{itemize}
    \item 6
    \begin{itemize}
        \item We have quantities $a,b$ with $a>b$. The two numbers are in the silver ratio if $\frac{b}{a}=\frac{a}{b+2a}$
        \item
        \begin{flalign*}
            &\phi = \frac{b}{a} = \frac{a}{b+2a} \\
            &b = a\phi \\
            &\phi = \frac{a}{a\phi+2a} = \frac{1}{\phi+2} \\
            &\phi^2 +2\phi -1 =0 \\
            &\phi = \frac{-2 \pm \sqrt{4-4(-1)(1)}}{2} =\frac{-2+2\sqrt{2}}{2} = 1+\sqrt{2}
        \end{flalign*}
    \end{itemize}
    \item 7
    \begin{itemize}
        \item Reverse Triangle Inequality - $\forall u,v \in \mathbb{R}, \vert u-v \vert \ge \vert \vert u \vert - \vert v \vert \vert$
        \item Triangle Inequality - $\forall u,v \in \mathbb{R}, \vert u+v \vert \le \vert u \vert + \vert v \vert$
        \item Proof for Triangle Inequality:
        \begin{flalign*}
            &- \vert x \vert \le x \le \vert x \vert \wedge - \vert y \vert \le y \le \vert y \vert \\
            \Rightarrow &- \vert x \vert - \vert y \vert \le x+y \le \vert x \vert + \vert y \vert \\
            \Longleftrightarrow &-(\vert x \vert + \vert y \vert) \le x+y \le \vert x \vert + \vert y \vert \\
            \Longleftrightarrow &\vert x + y \vert \le \vert x \vert + \vert y \vert
        \end{flalign*}
        \item Proof for Reverse Triangle Inequality:
        \begin{flalign*}
            &\vert x + y \vert \le \vert x \vert + \vert y \vert \qquad \text{Triangle Inequality} \\
            & \vert u \vert = \vert (u-v) + v \vert \le \vert u-v \vert + \vert v \vert \qquad \text{Substitute in } x=u-v, y=v \\
            &\vert u \vert - \vert v \vert \le \vert u -v \vert  \qquad \text{Subtraction preserves inequality} \\ 
            & \vert v \vert = \vert (v-u) + u \vert \le \vert v-u \vert + \vert u \vert \qquad \text{Substitute in } x=v-u, y=v \\
            &\vert v \vert - \vert u \vert \le \vert v-u \vert = \vert u-v \vert \qquad \text{Subtraction preserves inequality} \\
            &-\vert u-v \vert \le -(\vert v \vert - \vert u \vert) = \vert u \vert - \vert v \vert \qquad \text{Multiplication by negative number flips inequality, Distributive law of multiplication} \\
            &- \vert u-v\vert \le \vert u \vert - \vert v \vert \le \vert u-v \vert \qquad \text{Put equations (3) and (6) together} \\
            &\vert \vert u \vert - \vert v \vert \vert \le \vert u-v \vert \qquad \text{Definition of AVF}
        \end{flalign*}
    \end{itemize}

    \item 8a
    \begin{itemize}
        \item Let $a,b \in \mathbb{R}^+$ and $QM = \sqrt{\frac{a^2+b^2}{2}}, \quad AM = \frac{a+b}{2}, \quad GM =\sqrt{ab}, \quad HM = \left( \frac{a^{-1}+b^{-1}}{2} \right)^{-1}$
        \item $HM=\frac{2}{\frac{1}{a}+\frac{1}{b}}=\frac{2ab}{a+b} = \frac{2GM^2}{a+b} \Rightarrow GM^2 = \frac{a+b}{2}HM = AM \cdot HM$
        \item $\log(AM) = \log(\frac{a+b}{2})$
        \item $\log(GM) = \log(\sqrt{ab}) = \frac{1}{2}(\log{a} + \log{b})$
        \item By Jensen's Inequality, if $x \in \mathbb{R}^+$, since if $f(x)=\log(x)$, $f''(x)=-\frac{1}{x^2} \le 0$ (i.e., log is a concave function for $x \in \mathbb{R}^+$), $\log(E[x]) \ge E[\log(x)]$
        \begin{itemize}
            \item Therefore, in $x \in \mathbb{R}^+$, $\log(\frac{a+b}{2}) \ge \frac{1}{2} (\log(a) + \log(b))$
            \item Because the $\exp(x)$ function is a monotonically increasingly function in $x \in \mathbb{R}$, raising $e$ to the power of both sides preserves the inequality and we get $\frac{a+b}{2} \ge \sqrt{ab} \Rightarrow AM \ge GM$
        \end{itemize}
    \end{itemize}
    \item 8b
    \begin{itemize}
        \item Since $AM \ge GM$, $0 < \frac{AM}{GM} \le 1$. Dividing both sides by $GM$ in the equation that we derived for before, $GM^2 \frac{1}{GM} = GM = AM \cdot HM \frac{1}{GM} = \frac{AM}{GM} HM$, since $0 < \frac{AM}{GM} \le 1$, $GM \ge HM$
    \end{itemize}
    \item 8c
    \begin{itemize}
        \item Finally, we have $QM = \sqrt{\frac{a^2+b^2}{2}}$
        \item $QM^2 = \frac{a^2+b^2}{2}$
        \item $AM^2=\frac{(a+b)^2}{4}=\frac{a^2+b^2+2ab}{4}$
        \item Proposition: $QM(a,b)^2 \ge AM(a,b)^2$ for $a,b \in \mathbb{R}^+$
        \begin{itemize}
            \item Proof by Contradiction: Assume $AM^2 \ge QM^2$
            \item $\frac{a^2+b^2+2ab}{4} \ge \frac{a^2+b^2}{2}$
            \item $a^2+b^2+2ab \ge 2(a^2+b^2)$
            \item $2ab \ge a^2+b^2$
            \item $0 \ge a^2+b^2-2ab$
            \item $0 \ge (a-b)^2$
            \item But since $a,b \in \mathbb{R}^+$, $(a-b) \in \mathbb{R}$ and any real number squared is always greater than or equal to zero
            \item Therefore, our initial assumption is false and $QM^2 \ge AM^2$
        \end{itemize}
        \item Since $AM, QM \in \mathbb{R}^+$ for $a,b \in \mathbb{R}^+$, $QM^2 \ge AM^2 \Rightarrow QM \ge AM$
        \item Finally, putting everything together, we get $QM \ge AM \ge GM \ge HM$
    \end{itemize}
    \item 8d
    \begin{itemize}
        \item Proposition: For $a,b \in \mathbb{R}^+$, $QM=AM=GM=HM \Longleftrightarrow a=b$
        \begin{itemize}
            \item $a=b \Rightarrow QM=AM=GM=HM$
            \begin{itemize}
                \item $QM=\sqrt{\frac{a^2+b^2}{2}}=\sqrt{\frac{a^2+a^2}{2}}=\sqrt{\frac{2a^2}{2}}=\sqrt{a^2}=a=\frac{2a}{2}=\frac{a+a}{2}=\frac{a+b}{2}=AM$
                \item $QM=\sqrt{\frac{a^2+b^2}{2}}=\sqrt{\frac{a^2+a^2}{2}}=\sqrt{\frac{2a^2}{2}}=\sqrt{a^2}=\sqrt{ab}=GM$
                \item $AM=\frac{a+b}{2} = \frac{a+a}{2}=\frac{(a+a)\frac{1}{a}}{2\frac{1}{a}}=\frac{2}{\frac{2}{a}}=\left(\frac{\frac{2}{a}}{2} \right)^{-1}=\left(\frac{\frac{1}{a}+\frac{1}{a}}{2} \right)^{-1}=\left(\frac{a^{-1}+a^{-1}}{2} \right)^{-1}=\left(\frac{a^{-1}+b^{-1}}{2} \right)^{-1}=HM$
                \item $QM=AM=GM=HM$
            \end{itemize}
            \item $QM=AM=GM=HM \Rightarrow a=b$
            \begin{itemize}
                \item
                \begin{flalign*}
                    QM=AM &\Rightarrow \sqrt{\frac{a^2+b^2}{2}}=\frac{a+b}{2} \\
                    &\Rightarrow \frac{a^2+b^2}{2} = \left( \frac{a+b}{2} \right)^2 = \frac{a^2+b^2+2ab}{4} \\
                    &\Rightarrow 2(a^2+b^2) = a^2+b^2+2ab \\
                    &\Rightarrow a^2+b^2 = 2ab \\
                    &\Rightarrow a^2+b^2-2ab=0 \\
                    &\Rightarrow (a-b)^2 = 0 \\
                    &\Rightarrow a-b = 0 \\
                    &\Rightarrow a=b
                \end{flalign*}
            \item We have proven $a=b \Rightarrow QM=AM=GM=HM$ and $QM=AM=GM=HM \Rightarrow a=b$, so we have proven $QM=AM=GM=HM \Longleftrightarrow a=b$
            \end{itemize}
        \end{itemize}
    \end{itemize}
\end{itemize}

\end{document}
