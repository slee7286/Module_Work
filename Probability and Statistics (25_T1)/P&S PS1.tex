\documentclass{article}
\usepackage{graphicx} % Required for inserting images
\usepackage[a4paper, left=2cm, right=2cm, top=3cm, bottom=3cm]{geometry}
\usepackage{amsmath}
\usepackage{amssymb}


\title{Probability and Statistics PS1}
\author{Siheon Lee}
\date{October 2025}

\begin{document}

\maketitle

\section{Question 1}

\section{Question 2}
\begin{itemize}
    \item $P(A \cap B) = P(A)+P(B)-P(A \cup B) = 0.3+0.5-0.7=0.1$
    \item $P(A^C \cup B^C) = P((A \cap B)^C) = 1- 0.1 = 0.9$
    \item $P(A^C \cap B)= P(B) - P(A \cap B) = 0.5 - 0.1 = 0.4$
\end{itemize}

\section{Question 3}
\begin{itemize}
    \item a
    \begin{flalign*}
        &\text{The number of ways to form Team 1} \\
        &\binom{12}{2} = \frac{12!}{10! \cdot 2!} = 66 \\
        &\text{The number of ways to form Team 2} \\
        &\binom{10}{5} = \frac{10!}{5! \cdot 5!} = 252 \\
        &\text{The number of ways to form Team 3} \\
        &\binom{5}{5} = 1 \\
        &\text{Number of variations of teams} \\
        &2! = 2 \\
        &\text{Answer} \\
        & \frac{66 \cdot 252}{2} = 8316
    \end{flalign*}
    
    \item b
    \begin{flalign*}
        &\text{The number of ways to form Team 1} \\
        &\binom{12}{4} = \frac{12!}{8! \cdot 4!} = 495 \\
        &\text{The number of ways to form Team 2} \\
        &\binom{8}{4} = \frac{8!}{4! \cdot 4!} = 70 \\
        &\text{The number of ways to form Team 3} \\
        &\binom{4}{4} = 1 \\
        &\text{Number of variations of teams} \\
        &3! = 6 \\
        &\text{Answer} \\
        & \frac{495 \cdot 70}{6} = 5775
    \end{flalign*}
\end{itemize}

\section{Question 4}
\begin{itemize}
    \item (a) \[ \begin{array}{cc}
        \text{Summation to 9} & \text{Summation to 10}  \\
        \hline
        (1,2,6) = 3! = 8 & (1,4,5) = 3! = 8 \\
        (1,3,5) = 3! = 8 & (1,3,6) = 3! = 8 \\ 
        (1,4,4) = \binom{3}{2} = \frac{3!}{2! \cdot 1!} = 3 & (2,2,6) = \binom{3}{2} = \frac{3!}{2! \cdot 1!} = 3 \\ 
        (2,3,4) = 3! = 8 & (2,3,5) = 3! = 8 \\ 
        (2,2,5) = \binom{3}{2} = \frac{3!}{2! \cdot 1!} = 3 & (2,4,4) = \binom{3}{2} = \frac{3!}{2! \cdot 1!} = 3 \\ 
        (3,3,3) = 1 & (3,3,4) = \binom{3}{2} = \frac{3!}{2! \cdot 1!} = 3 \\ 
        \hline
        \text{Sum } = 23 & \text{Sum } = 25
    \end{array}\]
    \item There are 23 permutations of three dice that sum to 9.
    \item (b) There are 25 permutations of three dice that sum to 10.
    \item (c) There are $\binom{6}{3} = \frac{6!}{3! \cdot 3!} = 120$ permutations of three dice. Galileo's solution was to eliminate $(3,3,3)$ and $(3,3,4)$ from the gambling to make the probabilities equal.
\end{itemize}

\section{Question 5}
\begin{itemize}
    \item $P(\text{breakdown}) = 0.8 \cdot 0.05 + 0.1 \cdot 0.1 + 0.1 \cdot 0.9 = 0.04 + 0.01 + 0.09 = 0.14$
    \item There is a 14\% probability that the car is going to break down in the road trip.
\end{itemize}

\end{document}
